\section{Supply Chain Management} \label{sec:General}

There are billions of products being manufactured every day through complex supply chains that can extend to all parts of the world. However, tracing good flows from harvesting and manufacturing to the final consumer is hard \cite{galvez2018future}. Traceability is one of the key challenges encountered in the business world. Supply chain's transparency and end-to-end visibility can help shape product, raw material, test control, and end product flow\cite{tian2017supply}. Traceability systems typically store information in standard databases controlled by service providers. This centralized data storage becomes a single point of failure and tampering risks. As a consequence, these systems result in trusting problems, such as fraud, corruption, and tampering. Likewise, as a single point of failure, a centralized system is vulnerable to collapse \cite{tian2017supply}.

Nowadays, Blockchain presents a whole new approach based on decentralization, enabling end-to-end traceability, allowing consumers to access the asset's history of these products through a software application \cite{galvez2018future}. Supply chain management (SCM) requires to control who can write and read data to/from the Blockchain. In order to do that, the first step is identity. In the SCM context, the peers are known and the system needs to know who a user is, to define rules about what data they can commit, and what data they can consume from the ledger. So, in a corporate case scenario, Blockchain for the business, Blockchain for supply value chains, a private Blockchain provides this needed characteristic.