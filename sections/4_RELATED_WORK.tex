\section{Related Work} \label{sec:RelatedWork}
In order to solve some problems with Supply Chain traceability centralized, monopolistic, and asymmetric systems, many internet of things (IoT) technologies has been applied. However, these technologies do not guarantee that the information shared by supply chain members in the traceability systems can be trusted \cite{tian2017supply}.

In \cite{tian2017supply}, it is proposed a system that combines HACCP (Hazard Analysis and Critical Control Point, a food safety protocol), Blockchain and IoT in order to provide food safety traceability. Each supplu chain member can add, update and check the information about the product on the Blockchain as long as they register as a user in the system. Each product has also a unique digital cryptographic identifier that connects the physical items to their virtual identity in the system. This virtual identity can be seen as the product information profile.

The Everledger Diamonds project provides a Blockchain based solution to facilitate tracking from mine to consumer, enabling easier compliance against increasingly strict measures from diamonds produced \cite{crosby2016blockchain}.

These projects are focused on specific products only and are closed projects. Still, there is a general lack of standards for implementation of a Blockchain approach for traceability. A Blockchain must be universal and adaptable to specific situations \cite{valenta2017comparison}. In addition, the need to agree on a particular type of Blockchain to be used puts the parties under pressure. 

Our work is intended to provide a Blockchain based framework in order to facilitate the development of applications for traceability in supply chain management.