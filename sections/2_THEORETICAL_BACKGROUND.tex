\section{Blockchain} \label{sec:Theoretical}

Blockchain can be considered as a public ledger, in which all committed transactions are stored in a block chain \cite{zheng2016blockchain}. For \cite{swan2015blockchain}, Blockchain is in a position to become the fifth disruptive computing paradigm after mainframes, PCs, Internet and mobile/ social networks. Blockchain technology has critical features, such as decentralization, persistence, anonymity and auditability. Also, Blockchain can function in a decentralized environment that is activated by the integration of several key technologies such as cryptographic hash, digital signature and distributed consensus engine, significantly save the cost and improve efficiency \cite{zheng2016blockchain}.

\subsection{Public Blockchain Versus Private Blockchain}\label{sec:versus}
On a public blockchain, any person can participate without a specific identity. They can be audited by anyone, and each node has as much transmission power as any other. For a transaction to be considered valid, it must be authorized by all nodes constituents via the consensus process. As long as each node meets protocol-specific stipulations, their transactions can be validated and thus added to the chain \cite{greve2018blockchain}. Private blockchains, on the other hand, perform a blockchain between a set of known and identified participants. A private blockchain provides a way to protect the interactions between a group of entities that have a common goal but that don't totally trust each other, like companies that trade funds, assets or information \cite{swan2015blockchain}.

%%%%%%%%%%%%%%%%%%%%%%%%%%%%%%%%%%%%%%%%%%%%%%
\subsection{Smart Contracts}\label{sec:smartContracts}
A smart contract is a computerized transaction protocol that executes the terms of a contract \cite{szabo1997idea}. The term smart contract (SC) means: “an internal transaction protocol format that executes the terms of a contract. Their overall goals are ensure common contractual conditions, minimize malicious and accidental exceptions and the need for reliable intermediaries. Related economic objectives include reducing fraud losses, arbitration and execution costs, and other transaction costs.” \cite{szabo1997idea}.

%%%%%%%%%%%%%%%%%%%%%%%%%%%%%%%%%%%%%%%%%%%%%%